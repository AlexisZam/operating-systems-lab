\documentclass[12pt,titlepage]{article}

\usepackage[utf8]{inputenc}
\usepackage[english,greek]{babel}

\newcommand{\en}{\textlatin}

\usepackage[a4paper]{geometry}
\usepackage{listings}
\usepackage{xcolor}

\usepackage{underscore}

\title{Εργαστήριο Λειτουργικών Συστήματων\\
1η Άσκηση}
\author{Αλέξιος Ζαμάνης\\
03115010\\
Μιχαήλ Μεγγίσογλου\\
03115014}

\lstset{
	language=C,
	basicstyle=\ttfamily,
	breaklines=true,
	tabsize=4,
	showstringspaces=false,
	keywordstyle=\color{blue},
	stringstyle=\color{red}
}

\begin{document}

\maketitle

\section{Δοκιμασίες}

\begin{enumerate}

\setcounter{enumi}{-1}

\item

Πρόβλημα:

\latintext
\begin{lstlisting}[language=bash]
$ strace ./riddle
openat(AT_FDCWD, ".hello_there", O_RDONLY) = -1 ENOENT (No such file or directory)
\end{lstlisting}
\greektext

Λύση:

\latintext
\begin{lstlisting}[language=bash]
$ touch .hello_there
\end{lstlisting}
\greektext

Σχόλια: Δεν υπάρχει το αρχείο, ούτε το ανοίγουμε με \en{O_CREAT}, οπότε το δημιουργούμε.

\item

Πρόβλημα:

\latintext
\begin{lstlisting}[language=bash]
$ strace ./riddle
openat(AT_FDCWD, ".hello_there", O_WRONLY) = 4
\end{lstlisting}
\greektext

Λύση:

\latintext
\begin{lstlisting}[language=bash]
$ chmod -w .hello_there
\end{lstlisting}
\greektext

Σχόλια: Αφαιρούμε το δικαίωμα εγγραφής από το αρχείο.

\item

Πρόβλημα:

\latintext
\begin{lstlisting}[language=bash]
$ strace ./riddle
rt_sigaction(SIGALRM, {sa_handler=0x55e2e8605d40, sa_mask=[ALRM], sa_flags=SA_RESTORER|SA_RESTART, sa_restorer=0x7f0e02f31f20}, {sa_handler=SIG_DFL, sa_mask=[], sa_flags=0}, 8) = 0
rt_sigaction(SIGCONT, {sa_handler=0x55e2e8605d40, sa_mask=[CONT], sa_flags=SA_RESTORER|SA_RESTART, sa_restorer=0x7f0e02f31f20}, {sa_handler=SIG_DFL, sa_mask=[], sa_flags=0}, 8) = 0
alarm(10) = 0
pause() = ? ERESTARTNOHAND (To be restarted if no handler)
--- SIGALRM {si_signo=SIGALRM, si_code=SI_KERNEL} ---
rt_sigreturn({mask=[]}) = -1 EINTR (Interrupted system call)

\end{lstlisting}
\greektext

Λύση:

\latintext
\begin{lstlisting}[language=bash]
$ ./riddle
\end{lstlisting}
\greektext

\latintext
\begin{lstlisting}[language=bash]
$ pkill -SIGCONT riddle
\end{lstlisting}
\greektext

Σχόλια: Ορίζονται δύο \en{handlers}, ένας για \en{SIGALRM} και ένας για \en{SIGCONT}. Μέσα σε 10 δευτερόλεπτα στέλνουμε \en{SIGCONT} στη διεργασία, πριν τη σκοτώσει ο \en{handler} της \en{SIGALRM}.

\item

Πρόβλημα:

\latintext
\begin{lstlisting}[language=bash]
$ ltrace ./riddle
getenv("ANSWER") = nil
\end{lstlisting}
\greektext

Λύση:

\latintext
\begin{lstlisting}[language=bash]
$ export ANSWER=42
\end{lstlisting}
\greektext

Σχόλια: Ορίζουμε τη μεταβλητή περιβάλλοντος \en{ANSWER} και της δίνουμε την τιμη 42 (προφανώς).

\item

Πρόβλημα:

\latintext
\begin{lstlisting}[language=bash]
$ strace ./riddle
openat(AT_FDCWD, "magic_mirror", O_RDWR) = -1 ENOENT (No such file or directory)
\end{lstlisting}
\greektext

Λύση:

\latintext
\begin{lstlisting}[language=bash]
$ mkfifo magic_mirror
\end{lstlisting}
\greektext

Σχόλια: Δημιουργούμε ένα \en{named pipe} με το όνομα \en{magic_mirror}, ώστε η διεργασία να μπορεί με \en{FIFO} τρόπο να διαβάζει ό,τι γράφει.

\item

Πρόβλημα:

\latintext
\begin{lstlisting}[language=bash]
$ strace ./riddle
fcntl(99, F_GETFD) = -1 EBADF (Bad file descriptor)
\end{lstlisting}
\greektext

Λύση:

\latintext
\begin{lstlisting}[language=bash]
$ exec 99<>foo
\end{lstlisting}
\greektext

Σχόλια: Δημιουργούμε και ανοίγουμε ένα αρχείο ονόματι \en{foo} με περιγραφητή 99. Αυτός κληρονομείται από τη \en{riddle}.

\item

Πρόβλημα:

\latintext
\begin{lstlisting}[language=bash]
$ strace -f ./riddle
[pid 545] read(33, <unfinished ...>
[pid 545] <... read resumed> 0x7ffe758cbcdc, 4) = -1 EBADF (Bad file descriptor)
[pid 544] write(34, "\0\0\0\0", 4 <unfinished ...>
[pid 544] <... write resumed> ) = -1 EBADF (Bad file descriptor)
$ strace -f ./challenge6
[pid 623] read(33, <unfinished ...>
[pid 623] <... read resumed> 0x7ffe758cbcdc, 4) = -1 EBADF (Bad file descriptor)
[pid 622] write(34, "\0\0\0\0", 4 <unfinished ...>
[pid 622] <... write resumed> ) = -1 EBADF (Bad file descriptor)
[pid 622] read(53, <unfinished ...>
[pid 622] <... read resumed> 0x7fffa6953b4c, 4) = -1 EBADF (Bad file descriptor)
[pid 623] write(54, "\1\0\0\0", 4) = -1 EBADF (Bad file descriptor)
\end{lstlisting}
\greektext

Λύση:

\latintext
\begin{lstlisting}[language=bash]
$ ./challenge6
\end{lstlisting}
\greektext

Σχόλια: Ανοίγουμε 2 \en{pipes} και αντιγράφουμε τα άκρα τους στους ζητούμενους περιγραφητές, ήτοι 33, 34, 53 και 54.

\item

Πρόβλημα:

\latintext
\begin{lstlisting}[language=bash]
$ strace ./riddle
lstat(".hey_there", 0x7fff5fed1390) = -1 ENOENT (No such file or directory)
\end{lstlisting}
\greektext

Λύση:

\latintext
\begin{lstlisting}[language=bash]
$ ln .hello_there .hey_there
\end{lstlisting}
\greektext

Σχόλια: Δημιουργούμε ένα σύνδεσμο στο αρχείο \en{.hello_there} ονόματι \en{.hey_there}.

\item

Πρόβλημα:

\latintext
\begin{lstlisting}[language=bash]
$ strace ./riddle
openat(AT_FDCWD, "bf00", O_RDONLY) = -1 ENOENT (No such file or directory)
openat(AT_FDCWD, "bf00", O_RDONLY) = -1 ENOENT (No such file or directory)
openat(AT_FDCWD, "bf00", O_RDONLY) = -1 ENOENT (No such file or directory)
openat(AT_FDCWD, "bf00", O_RDONLY) = -1 ENOENT (No such file or directory)
openat(AT_FDCWD, "bf00", O_RDONLY) = -1 ENOENT (No such file or directory)
openat(AT_FDCWD, "bf00", O_RDONLY) = -1 ENOENT (No such file or directory)
openat(AT_FDCWD, "bf00", O_RDONLY) = -1 ENOENT (No such file or directory)
openat(AT_FDCWD, "bf00", O_RDONLY) = -1 ENOENT (No such file or directory)
openat(AT_FDCWD, "bf00", O_RDONLY) = -1 ENOENT (No such file or directory)
openat(AT_FDCWD, "bf00", O_RDONLY) = -1 ENOENT (No such file or directory)
\end{lstlisting}
\greektext

Λύση:

\latintext
\begin{lstlisting}[language=bash]
$ for i in {0..9}
do
	ln -s /dev/urandom bf0${i}
done
\end{lstlisting}
\greektext

Σχόλια: Δημιουργούμε συμβολικούς συνδέσμους στο \en{urandom} ονόματι \en{.bf0\{0..9\}}, οπότε αυτό επιστρέφει τυχαία δεδομένα για ανάγνωση.

\item

Πρόβλημα:

\latintext
\begin{lstlisting}[language=bash]
$ strace ./riddle
connect(4, {sa_family=AF_INET, sin_port=htons(49842), sin_addr=inet_addr("127.0.0.1")}, 16) = -1 ECONNREFUSED (Connection refused)
\end{lstlisting}
\greektext

Λύση:

\latintext
\begin{lstlisting}[language=bash]
$ nc -l localhost 49842
How much is 16054 + 1? 16055
\end{lstlisting}
\greektext
\latintext
\begin{lstlisting}[language=bash]
$ ./riddle
\end{lstlisting}
\greektext

Σχόλια: Ανοίγουμε μια σύνδεση \en{TCP} στο \en{localhost} με τη ζητούμενη θύρα και ακούμε σε αυτήν για την επερώτηση του προγράμματος. Έπειτα απαντάμε.

\item

Πρόβλημα:

\latintext
\begin{lstlisting}[language=bash]
$ strace ./riddle
openat(AT_FDCWD, "secret_number", O_RDWR|O_CREAT|O_TRUNC, 0600) = 4
unlink("secret_number") = 0
write(4, "The number I am thinking of righ"..., 4096) = 4096
mmap(NULL, 4096, PROT_READ|PROT_WRITE, MAP_SHARED, 4, 0) = 0x7f1c5d832000
close(4)= 0
\end{lstlisting}
\greektext

Λύση:

\latintext
\begin{lstlisting}[language=bash]
$ touch secret_number
$ ./challenge10
\end{lstlisting}
\greektext

Σχόλια: Δημιουργούμε το αρχείο \en{secret_number}. Έπειτα το πρόγραμμα μας το ανοίγει, ώστε να μη διαγραφεί με το \en{unlink}. To \en{riddle} γράφει στο αρχείο. Εμείς το διαβάζουμε ύστερα από 1 δευτερόλεπτο και το κλείνουμε. Έπειτα γράφουμε την απάντηση.

\item

Σχόλια: Ακριβώς όπως η προηγούμενη.

\item

Πρόβλημα:

\latintext
\begin{lstlisting}[language=bash]
$ strace ./riddle
openat(AT_FDCWD, "/tmp/riddle-0j0oR8", O_RDWR|O_CREAT|O_EXCL, 0600) = 4
ftruncate(4, 4096) = 0
mmap(NULL, 4096, PROT_READ|PROT_WRITE, MAP_SHARED, 4, 0) = 0x7fe7b419c000
\end{lstlisting}
\greektext

Λύση:

\latintext
\begin{lstlisting}[language=bash]
$ ls /tmp | grep riddle
$ ./challenge12 A /tmp/riddle-0j0oR8
\end{lstlisting}
\greektext

Σχόλια: Τρέχουμε το \en{riddle}. Μέσα σε λίγα δευτερόλεπτα βρίσκουμε το όνομα του αρχείου που δημιούργησε, τρέχουμε το πρόγραμμά μας με παραμέτρους το ζητούμενο γράμμα και το όνομα, και αυτο γράφει στο εν λόγω αρχείο το γράμμα στο σωστό \en{offset}.

\item

Πρόβλημα:

\latintext
\begin{lstlisting}[language=bash]
$ strace ./riddle
openat(AT_FDCWD, ".hello_there", O_RDWR|O_CREAT, 0600) = -1 EACCES (Permission denied)
$ strace ./riddle
ftruncate(4, 16384) = 0
\end{lstlisting}
\greektext

Λύση:

\latintext
\begin{lstlisting}[language=bash]
$ chmod +w .hello_there
$ ./challenge13
\end{lstlisting}
\greektext

Σχόλια: Επαναφέρουμε το δικαίωμα εγγραφής στο αρχείο \en{.hello_there}. Το πρόγραμμά μας τρέχει το \en{riddle}, που μικραίνει το μέγεθός του αρχείου, και το επαναφέρει στην αρχική τιμή του.

\item

Πρόβλημα:

\latintext
\begin{lstlisting}[language=bash]
$ strace ./riddle
getpid() = 7291
\end{lstlisting}
\greektext

Λύση:

\latintext
\begin{lstlisting}[language=bash]
$ ./challenge14
\end{lstlisting}
\greektext

Σχόλια: Το πρόγραμμά μας κάνει \en{fork} το \en{riddle} μέχρι να πετύχει το σωστό \en{pid}.

\end{enumerate}

\section{Παράρτημα: Πηγαίος κώδικας}

\latintext

\subsection{\textlatin{challenge6.c}}

\lstinputlisting{../riddle/challenge6.c}

\subsection{\textlatin{challenge10.c}}

\lstinputlisting{../riddle/challenge10.c}

\subsection{\textlatin{challenge12.c}}

\lstinputlisting{../riddle/challenge12.c}

\subsection{\textlatin{challenge13.c}}

\lstinputlisting{../riddle/challenge13.c}

\subsection{\textlatin{challenge14.c}}

\lstinputlisting{../riddle/challenge14.c}

\greektext

\end{document}