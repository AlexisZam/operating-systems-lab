\documentclass[12pt,titlepage]{article}

\usepackage[utf8]{inputenc}
\usepackage[english,greek]{babel}

\newcommand{\en}{\textlatin}

\usepackage[a4paper]{geometry}

\usepackage{listings}
\usepackage[dvipsnames]{xcolor}

\usepackage{underscore}

\title{Εργαστήριο Λειτουργικών Συστήματων\\
3η Άσκηση}
\author{Αλέξιος Ζαμάνης\\
03115010\\
Μιχαήλ Μεγγίσογλου\\
03115014}

\lstset{
	language=C,
	basicstyle=\ttfamily,
	breaklines=true,
	tabsize=4,
	showstringspaces=false,
	commentstyle=\color{ForestGreen},
	keywordstyle=\color{blue},
	numberstyle=\color{gray},
	stringstyle=\color{purple}
}

\begin{document}

\maketitle

\section{Σχόλια}

\subsection{\en{Chat}}

Η κατασκευή του \en{Chat} διαιρέθηκε, όπως προτεινόταν, σε δύο φάσεις: μία για την επικοινωνία \en{client-server} μέσω \en{BSD sockets} και μία για την κρυπτογράφηση αυτής μέσω \en{Cryptodev}.

Ως προς το κομμάτι της επικοινωνίας, αυτή ξεκινά από τον \en{server}, που ανοίγει ένα ορισμένο \en{TCP socket}, στο οποίο ακούει για αιτήματα. Εν συνεχεία, συνδέεται ο \en{client} στο γνωστό \en{socket}, και δεδομένης επιτυχούς χειραψίας, ο \en{server} αποδέχεται την επικοινωνία και η συζήτηση ξεκινά.

Η συζήτηση καθεαυτή αποτελείται από έναν αέναο βρόχο εναλλασσόμενων εγγραφών και αναγνώσεων, ήτοι ο εκάστοστε χρήστης είτε γράφει είτε περιμένει μπλοκάροντας για να διαβάσει. Ο βρόχος και επομένως η συζήτηση λήγει με αποστολή του μηνύματος \en{bye} από εναν εκ των δύο συνομιλητών. Στην πραγματικότητα ο αποστολέας του μηνύματος λήξης αποστέλει \en{shutdown} με όρισμα \en{SHUT_WR}, ήτοι πως παύει να γράφει, προκαλώντας εμμέσως το κλείσιμο του \en{socket} σε αμφότερες τις πλευρές.

Αξίζει να σημειωθεί η χρήση των συναρτήσεων \en{insist_read} και \en{insist_write}, που επιβάλλουν την αποστολή ολόκληρου του μηνύματος και την πλήρη ανάγνωσή του, ώστε να διατηρείται η εμπερία του \en{chat} από τους χρήστες.

Όσον αφορά το κομμάτι της κρυπτογράφησης, αυτό προστέθηκε εμβόλιμα στις προαναφερθείσες λειτουργίες. Συγκεκριμένα, πριν την έναρξη του ατέρμονος βρόχου ξεκινά ένα \en{session} με το \en{Cryptodev}, το οποίο λήγει ύστερα από το πέρας του βρόχου. Η ύπαρξη κοινού \en{session}, ήτοι με τις ίδιες κρυπτογραφικές σταθερές, είναι καίρια για την αποκρυπτογράφηση των δεδομένων. Στην ουσία τα μόνα δεδομένα εισόδου που μεταβάλλονται στο \en{cryptodev} είναι τα ίδια τα μηνύματα.

Η κρυπτογράφηση γίνεται πριν την αποστολή του μηνύματος και η αποκρυπτογράφηση κατόπιν της παραλαβής του. Στο τέλος του κρυπρογραφημένου μηνύματος προστίθεται ένας χαρακτήρας \en{newline}, που χρησιμοποιείται για την αναγνώριση της λήξης της αποστολής.

Κάθε κλήση για έναρξη ή λήξη \en{session}, καθώς και για (απο)κρυπτογράφηση αντιστοιχεί σε μια αντίστοιχει κλήση συστήματος στον οδηγό \en{Cryptodev}. Σε αυτή τη φάση της άσκησης οι κλήσεις εξυπηρετούνται απευθείας από τη συσκευή.

\subsection{\en{VirtIO}}

Στόχος του συγκεκριμένου πρωτοκόλλου είναι ο ορισμός μίας κοινής διεπαφής για επικοινωνία μεταξύ \en{host} και \en{guest} λειτουργικού συστήματος σε περιπτώσεις \en{paravirtualization}. Εν προκειμένω, το αξιοποιούμε ώστε να προωθήσουμε τις προαναφερθείσες κλήσεις συστήματος στο \en{Cryptodev} που εδράζει σε ένα \en{host} μηχάνημα, όταν αυτές αποστέλονται από τον \en{quest}.

Συγκεκριμένα, υλοποιούμε τις τρεις κλήσεις συστήματος, ήτοι τις τρεις περιπτώσεις της \en{ioctl}, ώστε να βάζουν τα δεδομένα τους, μαζί με την αντίστοιχη εντολή, σε μια δομή, ονόματι \en{scatter-gather list}, που, ως κοινή μνήμη μεταξύ πραγματικού και εικονικού μηχανήματος, μπορεί να γράφεται και να διαβάζεται από αμφότερα.

Σε κάθε κλήση της \en{ioctl}, ο \en{guest} φτιάχνει μια διαφορετική \en{sg} για το κάθε δεδομένο που θέλει να γράψει ή διαβάσει (μάλιστα πρώτα όλα τα προς ανάγνωση δεδομένα, και ύστερα όλα τα προς εγγραφή), τις προωθεί όλες μαζί σε μια δεδομένη ουρά που του έχει δοθεί, και κάνει \en{busy wait}, ενώ αναμένει τα αποτελέσματα. Κατόπιν λαμβάνει τα δεδομένα του \en{Cryptodev} και επιστρέφει αναλόγως στο χρήστη.

Κατά τη διάρκεια της αναμονής, ο \en{guest} προφυλάσσεται από έναν σημαφόρο, ουσιαστικά ένα \en{mutex}, που δεν επιτρέπει πρόσβαση από άλλες διεργασίες στην υπο χρησιμοποίηση ουρά. Η χρήση σημαφόρου, έναντι \en{spinlock} είναι προτιμητέα, καθώς τα δεύτερα μονοπωλούν το χρόνο της \en{CPU}, εν προκειμένω καθόλη την ανταλλαγή των δεδομένων. Μάλιστα, δεδομένου ότι ο κώδικάς μας πρόκειται να εκτελείται μόνο σε \en{process context}, η χρήση \en{spinlock} είναι περιττή, καθώς δεν υπάρχει η απαίτηση να μην μπορεί να κοιμηθεί.

Από την πλευρά του \en{host} μηχανήματος, και συγκεκριμένα στον πηγαίο κώδικα του \en{QEMU}, υλοποιείται η παραλαβή των δεδομένων από τις ουρές, σε πλήρη, ένα προς ένα αντιστοιχία με την σειρά τοποθέτησής τους από τον \en{guest}. Από εκεί εκτελούνται, ωσάν δια αντιπροσώπου, οι αντίστοιχες κλήσεις συστήματος στο \en{Cryptodev}. Εν τέλει, στέλνονται τα αντίστοιχα αποτελέσματα.

Κατά πλήρη αναλογία με τα παραπάνω, υλοποιούνται και οι συναρτήσεις \en{open} και \en{release}, που καλούνται στο άνοιγμα μίας συγκεκριμένης εικονικής συσκευής \en{Cryptodev} και στο κλείσιμό της αντίστοιχα. Εδώ δεσμεύονται ή απελευθερώνονται οι απαραίτητες δομές για την συσκευή και στέλνοται όπως πάνω στο \en{host} μηχάνημα. Σημειώνουμε πως η επιλογή της κατάλληλης συσκευής μέσω \en{inodes}, καθώς και οι απαραίτητες αρχικοποιήσεις και καταστροφές αυτών, γίνονται όπως στην προηγούμενη άσκηση, οπότε δε χρήζουν ιδιαίτερης μνείας.

\newpage

\section{Παράρτημα: Πηγαίος κώδικας}

\subsection{\en{Chat}}

\latintext

\subsubsection{mycrypto.h}

\lstinputlisting{../virtio-cryptodev-helpcode/chat/mycrypto.h}

\subsubsection{mysocket-common.h}

\lstinputlisting{../virtio-cryptodev-helpcode/chat/mysocket-common.h}

\subsubsection{mysocket-client.c}

\lstinputlisting{../virtio-cryptodev-helpcode/chat/mysocket-client.c}

\subsubsection{mysocket-server.c}

\lstinputlisting{../virtio-cryptodev-helpcode/chat/mysocket-server.c}

\subsection{\en{VirtIO}}

\subsubsection{crypto.h}

\lstinputlisting{../virtio-cryptodev-helpcode/virtio-cryptodev/guest/crypto.h}

\subsubsection{crypto-module.c}

\lstinputlisting{../virtio-cryptodev-helpcode/virtio-cryptodev/guest/crypto-module.c}

\subsubsection{crypto-chrdev.c}

\lstinputlisting{../virtio-cryptodev-helpcode/virtio-cryptodev/guest/crypto-chrdev.c}

\subsubsection{virtio-cryptodev.c}

\lstinputlisting{../virtio-cryptodev-helpcode/virtio-cryptodev/qemu-3.0.0/hw/char/virtio-cryptodev.c}

\greektext

\end{document}