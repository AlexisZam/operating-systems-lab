\documentclass[12pt,titlepage]{article}

\usepackage[utf8]{inputenc}
\usepackage[english,greek]{babel}

\usepackage[a4paper]{geometry}
\usepackage{listings}
\usepackage[dvipsnames]{xcolor}

\title{Εργαστήριο Λειτουργικών Συστήματων\\
2η Άσκηση}
\author{Αλέξιος Ζαμάνης\\
03115010\\
Μιχαήλ Μεγγίσογλου\\
03115014}

\lstset{
	language=C,
	basicstyle=\ttfamily,
	breaklines=true,
	numbers=left,
	tabsize=4,
	showstringspaces=false,
	commentstyle=\color{ForestGreen},
	keywordstyle=\color{blue},
	numberstyle=\color{gray},
	stringstyle=\color{purple}
}

\begin{document}

\maketitle

\section{Σχολιασμός των συναρτήσεων}

\subsection{\textlatin{lunix\_chrdev\_init}}

Ζητάμε από τον πυρήνα το \textlatin{major number} 60 για το \textlatin{character device} μας με όνομα \textlatin{Lunix:TNG}, μαζί με μία σειρά από \textlatin{minor numbers} για κάθε πιθανό ζεύγος αισθητήρα-μέτρησης. Αν μας δοθεί, καταχωρούμε το \textlatin{device} μας στο σύστημα και αυτό γίνεται άμεσα διαθέσιμο.

\subsection{\textlatin{lunix\_chrdev\_open}}

Παίρνουμε από το \textlatin{inode}, ήτοι το ζεύγος αισθητήρα-μέτρησης που μας ενδιαφέρει, το \textlatin{minor number} του και δεσμεύουμε-αρχικοποιούμε βάσει αυτού το πεδίo \textlatin{private\_data} που αντιστοιχεί στο εκάστοτε άνοιγμα αρχείου του νήματος εκτέλεσής μας. Στην ίδια δομή αρχικοποιούμε ένα \textlatin{mutex}, που θα χρησιμοποιήσουμε παρακάτω.

\subsection{\textlatin{lunix\_chrdev\_release}}

Στο κλείσιμο του \textlatin{instance} του ανοιχτού αρχείου που βλέπει το νήμα εκτέλεσης μας απελευθερώνομε τους πόρους που είχε δεσμεύσει το πεδίο \textlatin{private data}.

\subsection{\textlatin{lunix\_chrdev\_read}}

Εδώ βρίσκεται η καρδιά του οδηγού συσκευών μας. Το ενδιαφερόμενο για ανάγνωση νήμα εκτέλεσης κλειδώνει ένα \textlatin{mutex}, μεταφέρει στο χώρο χρήστη όσα δεδομένα ζήτησε να διαβάσει (ή, αν είναι λιγότερα, όσα είναι διαθέσιμα), τοποθετεί το \textlatin{file offset} στη σωστή θέση (ή το μηδενίζει, αν φτάσαμε στο τέλος των διαθέσιμων δεδομένων), ξεκλειδώνει το \textlatin{mutex} και επιστρέφει το πλήθος των δεδομένων που διάβασε. Με το \textlatin{mutex} αποφεύγουμε την περίπτωση ταυτόχρονης πρόσβασης στα δεδομένα, που θα οδηγούσε σε εσφαλμένη τελική τιμή στο \textlatin{offset}.

Στην περίπτωση που δεν υπάρχουν δεδομένα, το νήμα εκτέλεσης ζητά ατομικά την ανανέωση του \textlatin{buffer}, καλώντας την \textlatin{lunix\_chrdev\_state\_update}, την οποία θα συζητήσουμε παρακάτω. Αν αυτή επιστρέψει κατά σύμβαση \textlatin{-EAGAIN}, ήτοι δεν υπάρχουν νέα δεδομένα, ξεκλειδώνουμε το \textlatin{mutex}, και μπαίνουμε στην ουρά αναμονής που αντιστοιχεί στο ζεύγος αισθητήρα-μέτρησης που μας ενδιαφέρει, έως ότου ικανοποιηθεί η συνθήκη που περιγράφει η συνάρτηση \textlatin{lunix\_chrdev\_state\_needs\_refresh} (βλέπε παρκάτω). Όσο περιμένουμε, κοιμόμαστε, ώστε να μη σπαταλάμε το χρόνο του επεξεργαστή. Όταν πλέον ξυπνήσουμε, ξαναπαίρνουμε το κλείδωμα και επαναλαμβάνουμε το αίτημα ανανέωσης του \textlatin{buffer}.

Αξίζει να σημειωθεί ότι χρησιμοποιούμε τις \textlatin{interruptible} εκδοχές των κλειδωμάτων και της ουράς αναμονής, ώστε πιθανά \textlatin{interrupts}, π.χ. ένα σήμα \textlatin{SIGKILL} από το χρήστη, να μπορούν να ξυπνήσουν το νήμα εκτέλεσης που έχει κοιμηθεί. Σε διαφορετική περίπτωση θα παρουσιάζονταν έντονο \textlatin{latency} προς το μέρος του χρήστη.

\subsection{\textlatin{lunix\_chrdev\_state\_update}}

Η συνάρτηση αυτή καλείται από ένα νήμα εκτέλεσης που έχει λάβει το κλείδωμα. Σε πρώτο στάδιο κλειδώνει ένα \textlatin{spinlock}, καλεί τη \textlatin{lunix\_chrdev\_state\_needs\_refresh} (βλέπε παρακάτω), και ανάλογα με την απάντησή της, είτε αποθηκεύει τοπικά τα νέα δεδομένα, ανανεώνοντας παράλληλα τη χρονοσφραγίδα και αφήνοντας εν συνεχεία το \textlatin{spinlock}, είτε ξεκλειδώνει το \textlatin{spinlock} και επιστρέφει κατα σύμβαση \textlatin{-EAGAIN} στον καλούντα.

Η χρήση \textlatin{spinlock} προστατεύει τον αισθητήρα από ταυτόχρονες εγγραφές σε \textlatin{interrupt context}. Μάλιστα, η εκδοχή των \textlatin{spinlocks} που χρησιμοποιούμε απενεργοποιεί τα \textlatin{interrupts} στον τοπικό επεξαργαστή, αποθηκεύοντας παράλληλα την προηγούμενη κατάσταση (ήτοι αν ήταν ήδη απενεργοποιημένα), αποτρέποντας πιθανά αδιέξοδα.

Στο δεύτερο στάδιο δίνουμε στα \textlatin{raw} δεδομένα τις τιμές που αντιστοιχούν στο μέγεθος που περιγράφουν, βάσει λεξικού, και τα μετατρέπουμε σε πλασματική δεκαδική αναπαράσταση, αποθηκεύοντας παράλληλα τη χρονοσφραγίδα και τη μορφοποιημένη τιμή στο πεδίο \textlatin{private\_data} του ανοιχτού αρχείου. Η επιλογή χρήσης ακεραίων οφείλεται σε περιορισμούς που επιβάλλει το λειτουργικό συστημα, αναφορικά με τους καταχωρητές των πράξεων κινητής υποδιαστολής. 

\subsection{\textlatin{lunix\_chrdev\_state\_needs\_refresh}}

Εδώ γρήγορα ελέγχουμε αν η χρονοσφραγίδα που διαθέτουμε προηγείται χρονικά της τελευταίας ανανέωσης του ζεύγους αισθητήρα-μέτρησης που μας αφορά και επιστρέφουμε την αντίστοιχη αληθοτιμή.

\section{Παράρτημα πηγαίου κώδικα}

\subsection{\textlatin{lunix-chrdev.c}}

\latintext

\lstinputlisting{../lunix-tng-helpcode/lunix-chrdev.c}

\greektext

\end{document}